%---------------------- the documentclass -----------------------------------------------------------------------------------------------------------%
% using this command you confige the "look and feel" of your document.
% you can pass different options to configure common usecases
% the options are shown for clarity, but most of them are allready the default,
% so no need to pass them explicitly
\documentclass[
        faculty=f1,
        paper=a4,
        fontsize=11pt,
        fontfamily=sans-serif,
        language=english,
        parskip=never+,
        linespacing=single,
        twoside=true,
        todos=off,
        draft=false,
    ]{HsH-report}

%---------------------- the preamble ----------------------------------------------------------------------------------------------------------------%
% everything between `\documentclass' and `\begin{document}' is called the preamble. Here you configure all settings for your document.
% The `\documentclass' command is actually part of that configuration. Lets see what you could do here:

% ----- package loading -----
% first thing you do is declare all the packages you need for your document
% you can also pass options to this packages to configure their behavior
\usepackage{listings}  % for pretty-printing code snippets
\usepackage{soul}  % for strickesthough text
\usepackage{lipsum}  % for dummy text

% for some packages you also call some commands to configure them or your document
\lstMakeShortInline[language={[LaTeX]TeX}]|

% ----- document information -----
% In your preamble you also list your documents information and metadata. These will be used on the titlepage as well
% as being available throughout the document. Additionally, these documentclasses set up the resulting PDF file with
% the appropriate Metadata.
% You can just delete any of this commands or leave them empty if you don't need it for a project.
% See the following examples and what they create in the PDF file:
\author{
    Max Mustermann,
    Mira Musterfrau
} % the author and matrikelnr commands could also be on a single line, this is just more readable
\matrikelnr{
    1234567,
    9876543
}
\titlehead{Found on GitLab}
\subject{Example Project}
\title{How to write in Latex}
\subtitle{A helpful guide to get started and to show some common use cases}
\date{\st{01.01.2020}\\\today}
\professor{your Professor}
\keywords{some, informative, keywords}


% ----- document seperation -----
% If you split your document into seperate files using `\include',
% you can temporarily exclude not required files to save compiletime
% LATeX will still remember chapter-numbers, page-numbers and alike
% from the last run (but only if the temp files are still around)
% comment this in to use it:
%\includeonly{chap/startingAdocument}

%---------------------- beginning of document -------------------------------------------------------------------------------------------------------%
% Now that you are all set, let's begin with the actual content of the document.
% Don't forget the corresponding `\end{document}'!
\begin{document}

    % for longer documents it is custom to have different numbering until the first page of actuall content.
    % For that use this command to switch to Roman pagenumbers and turn off chapternumbers:
    \frontmatter

    % While you can of course create your own title-page, either with latex or externally, the easiest way is to use the build in command.
    % These classes redefine it to include the HsH-logo (depending on the chosen faculty) and to use the additional data provided in the preamble.
    % You can also use the optional argument to change the title-pages alignment to l,c or r:
    \maketitle[c]

    % this command is provided by these documentclasses. It creates a standard Text at the bottom of the page and a line to sign on for every author.
    % You are not restricted to this exact position and can use it where ever you want in your document, if you prefer it at the back, but it there.
    \declarationofauthorship

    % sometimes you are required to also create an abstract. Use this environment for that.
    % It will create a new page and a heading for you as well as indenting the whole text block a little.
    % if you have provided keywords, they will also be put at the end of the abstract.
    \begin{abstract}
		If you need an abstract for your document, you can write it wherever you see fit by using the |\begin{abstract}...\end{abstract}|
		environment, like demonstrated here. It acts as an unnumbered chapter. You can choose if you want it in the TOC using the
		|abstract=totoc| and |abstract=nottotoc| options of the documentclass.

		If you prefere your abstract to be on a clean page, you can use |\thispagestyle{plain}| to get only a page number or |\thispagestyle{empty}|
		to get nor header or footer.

		If you use the |\keywords{list, of, keywords}| command in your preamble, the given keywords will also be printed here. You may use
		|abstract=nokeywords| as a documentclass option to disable this.
	\end{abstract}

    % this command will create the table of contents (TOC).
    \tableofcontents

    % the following command is the counterpiece of the `\frontmatter' command.
    % It resets pagenumbers so that the next chapter is the first with actuall content.
    \mainmatter

    % now we can begin with the actuall relevant content
    % you could just put all commands and content here,
    % but for larger documents it makes sense to split each chapter into a seperate file.
    % NOTE: you can use the \includeonly{} in the preamble to temporarily only work on a
    % small subset of the document.
    %
    % We include this files here:
    \chapter{What is LaTeX} \label{chap: latex}
    So you decided to get stated with LaTeX. Great! So let's talk a bit about the basic concept and differences it comes with.

    \medskip
    Up to this point you probably used a Word Processor like MS Word. The kind of workflow you know from there is often referred to as \emph{What you
    see is what you get}. You see the exact final layout as you type it, press some colourful buttons to insert stuff and if it doesn't want to do
    something you need, you're screwed.

    LaTeX on the other hand falls into the category of \emph{What you see is what you mean}, which describes all forms of markup languages. This means
    you create your LaTeX document as a simple plain text file without any from of formatting and mix in a bunch of commands telling what you mean.
    For example: "This is supposed to be a chapter heading", "make this bold" or "insert an image here". This source file will than be passed to a
    document processor (the LaTeX program), which will, depending on its settings, create the document for you. The advantage is, that you can use the
    same markup with all sorts of formattings and target file types.

    This is why working with LaTeX will require some getting used to and you will find yourself wanting to compile every five seconds to see the
    document update. Try to restrain yourself and concentrate on writing. You will find yourself working much faster.

    \section{Following this document}
        To see how the LaTeX source code and the resulting PDF correspond, I recommend you open this documents source code and PDF file next to each
        other and scroll through them simultaneously.

        If you already have a working LaTeX setup, most editors support \emph{SyncTex}, which allows you to jump between source code and PDF file and
        vice versa. You have to compile yourself, which will create a file called |example.synctex.gz| in your project directory. Now you
        can |<CTRL>+Click| in the PDF and the corresponding line of source code will be highlighted.

        The shortcut to jump from the source code into the PDF will depend on your Editor, but for VS\;Code its
        |right Click|→|SyncTex from cursor| or |CTRL+ALT+J|.

    \section{Requirements to use LaTeX}
        As LaTeX files are just plain text file, you can edit them with any text editor (even windows notepad works, but that's just terrible).
        However, I would strongly recommend a more suitable editor. I use \href{https://code.visualstudio.com/}{Visual Studio Code} (which is a multi
        porpoise text editor that support all major programming languages) but you could also use something like
        \href{https://www.xm1math.net/texmaker/}{Texmaker}, which is an editor specifically for LaTeX. There is also the online editor
        \href{https://www.overleaf.com/}{Overfleaf}, which saves you the trouble of setting up your own LaTeX installation and provides everything you
        need in the cloud.

        \pagebreak\medskip
        As I have already mention above, you also need the LaTeX program. It comes bundled with packages and other additional software inside a
        Tex-distribution. There are two major ones, Texlive and MiKTeX. I recommend MiKTeX, but it essentially doesn't matter which one you choose.

        Once you have the distribution installed, test it by running |pdflatex --version| in any terminal windows and it should return you
        some information about the installed version and setup.

    \section{Running LaTeX}
        To create a PDF file from your LaTeX source code, you can always navigate to the project folder in a terminal window and run
        |pdflatex filename.tex|. However, if you have a decent editor installed, it will provide you with a button and do this for you.

        With these project files you also received a makefile, which demonstrates how to compile this example file successfully from the terminal. The
        README file also has some tips and information for you.

        If you use VS Code, this project also contains settings for LaTeX and recommended extensions. If you open the folder for the first time you
        will be asked if you want to install them and should than be able to compile this file.

    \section{LaTeX commands}
        Now lets look at the LaTeX command. Every one will begin with a |\| followed by a letters only command name, like this:
        |\command|. Most commands also accept input, which is put after it into curly brackets: |\command{argument}|. They can
        accept multiple arguments either in multiple sets of curly brackets or as a comma separated list, depending on the command.

        Some commands also accept optional arguments. These are passed inside square brackets between the command name and the curly brackets, like
        this: |\command[optional]{argument}|.

    \section{Getting more information}
        So what can you do if you get stuck or just want more information. The simple answer is: Google is your friend. Most questions have already
        been answered. For example on \href{https://tex.stackexchange.com/}{Tex Strackexchange}. Also, Overleaf has a great
        \href{https://www.overleaf.com/learn}{section for learning LaTex}.

        An of course you can always check the documentation, which you can find on \href{https://ctan.org/}{CTAN}.

	\chapter{Starting a new document} \label{chap: starting new}

	\chapter{Formatting text} \label{:chap formating}
    To begin I want to show you the basics of how to get text onto the page and structure it. You can also see how I created this exact text as an
    example.

    \section{Texts and paragraphs} \label{sec: text and par}
        Writing text for a LaTeX document is very easy. You just put the text in. LaTeX doesn't care about line breaks or whitespace, so its up to
        your preferences how the source code looks. You could just write the whole document as one super long line, but it makes sense to break it up
        and keep it readable. One common way is to put every sentence on a new line. Alternatively lots of editors can break lines for you after you
        reach a certain width (that's how this source code is formatted).

        LaTeX will automatically space and break your text to optimal use the available space while still looking good. You can however assist it when
        it struggles. Putting hyphens (-) into a word tells LaTeX to break it there. If it's a word you use a lot, you can use
        |\hyphenation{very-long-word}| in your preamble to tell LaTeX how to split it everywhere.

        Lots of examples will tell you that |\\| is a line break. While this is correct you shouldn't use it to break your text block and
        start a new line. A block of text is a paragraph and should be ended with the |\par| command. For ease of use LaTeX will
        automatically use this command if you leave a blank line. (see this source code)

        If you want to further separate paragraphs visually (when you finish a train of thought for example) you can use the commands
        |\smallskip|, |\medskip| and |\bigskip|

    \section{Headings} \label{sec: headings}
        The exact commands available will vary depending on you your documentclass, but they will always be a single command that excepts any text
        inside curly brackets, for example |\section{text}|. The different commands form a hierarchy you can nest into each other, keeping
        track of its parent element. That means you don't have to worry about any formatting or numbering, LaTeX will handle that for you.

        When using an \emph{article} documentclass, the commands available are |\section{}|, |\subsection{}| and
        |\subsubsection{}|. Should you need more nesting levels, you are usually overcomplicating things, but you could additionally use the
        |\paragraph{}| command, which gives you a slightly bigger, bold first word for your paragraph.

        The \emph{report} documentclass adds the additional command |\chapter{}| as the highest heading level. You can still use the previous
        three commands for the nested headings. A chapter automatically starts on a new page, so it should be at leas two pages long. You also get the
        command |\part{}|, which creates a separate page for the part's title. These should only be used in very long documents.

    \section{Text spacing} \label{sec: spacing}
        By default, these classes add no spacing between paragraph, but sometimes you want to visually enforce a breakpoint in your argumentation. For
        that you can add some space in between to paragraph by using one of the commands |\bigskip|, |\medskip| or
        |\smallskip|. How much space you want depends on your tase, but you should keep it consistent. Here is an example:

        \bigskip
        This text has a big space before it,

        \medskip
        Here I used just some medium spacing

        \smallskip
        and this is a small space.

    \section{Breaking pages} \label{sec: pagebreak}
        Sometimes you will find yourself in situations, where you don't like where LaTeX splits your text to the next page. So first, take some
        advice: Don't worry about it for now. Your text will probably change a few times before its final. Just leave it.

        If you are at the final stage, you can do a beautifying pass. Now you can use |\pagebreak| to tell LaTeX about better places to
        break the text.

        Should you still not be happy (this happens especially with multiple images/tables in close proximity) you most likely have to little text and
        should redesign your document. But if you absolutely want to print it that way, you can use |\clearpage| to force all
        figures/tables to be put onto the page and then start a new page.

        \medskip
        You might also need just a little more space only a page to just fit one more sentences. For that you can use the command
        |\enlargethispage{N\baselineskip}| with $N$ being the number of lines you need. Use this sparingly however, as the bottom margin is
        there for a reason and you shouldn't intrude on the footer too much.


    \section{Text styling} \label{sec: styling}
        When writing text, you will need to \emph{emphasize} certain parts of the text. The easiest way is to use the |\emph{}| command
        around you text. You can also nest it \emph{to \emph{emphasize} even more}.

        If you want to change to a specific font-type, you can do that like this:

        \smallskip
        \begin{tabular}{l l}
            |\underline{text}| & \underline{Underlined} \\
            |\textbf{text}| & \textbf{Bold Font} \\
            |\textii{text}| & \textit{Italic Font} \\
            |\textrm{text}| & \textrm{Roman Font} \\
            |\texttt{text}| & \texttt{Typewriter Font} \\
            |\textsc{text}| & \texttt{Small Caps Font} \\
        \end{tabular}

        \medskip
        You might also want to change your text colour, which is what the {color} package is for. It provides the
        |textcolor{colour}{text}| command, \textcolor{red}{which allows you} \textcolor{blue}{to change your text colour}.

        \section{Special characters} \label{sec: special-charaters}
            \subsection{LaTeX command characters}
                As in most programming languages, some characters are used for LaTeXes commands and can't be used in text directly. Here is a table
                explaining them all:

                \smallskip
                \begin{tabular}{l l l}
                    \emph{character} & \emph{special meaning} & \emph{how to get character} \\
                    \textbackslash & beginning of a command & |\textbackslash| \\
                    \{ and \} & denote a code block & |\{| and |\}|\\
                    \% & beginning of a comment & |\%| \\
                    \# & macro parameter character & |\#| \\
                    \$ & beginning/end of math mode & |\$| \\
                    \textasciitilde & non-breaking space & |\textasciitilde| \\
                    \emph{only inside math mode:} \\
                    $\_$ & subscript & |\_| \\
                    \textasciicircum & superscript & |\textasciicircum| \\
                \end{tabular}

            \subsection{Invisible characters}
                To properly typeset your text you may need a number of special characters under specific circumstances:

                \smallskip
                \begin{tabular}{l l l}
                    \emph{explanation} & \emph{command} & \emph{example} \\
                    non-breaking space & |~| & Max~Mustermann (Names shouldn't be broken) \\
                    3/4 non-breaking space & |\;| & 10\;000 (separate thousands)\\
                    medium non-breaking space & |\:| & z.\:B. (abbreviations) \\
                    1/2 non-breaking space & |\,| & 1\,V (number + unit) \\
                    normal space & |\space| & in case some command eats up all your space \\
                \end{tabular}

    % ATTENTION: you can NOT nest multiple `\inlcude' commands into each other.
    % You can use `\input' inside included files though


    % print list of figures and tables
    \listoffigures
    \listoftables
\end{document}
