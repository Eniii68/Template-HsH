\documentclass[	%----------------------Preamble---------------------------------------------------%
		fontsize=11pt,  % fontsize
		a4paper,	    % papersize
		twoside,		% double sided layout
		english,		% document language (also numberingsystem)
		%ngerman,		% document language (also numberingsystem)
		sans,			% font type (sans/roman)
		f1				% HsH facultie (f1-f5)
	]{HsH-report}		% documentclass

% INFO:
% this file will produce a few warnings, even when working correctly:
% - using fallback Bibtex backend, cased because of the older backend, which may not support all features
% - Bad type area setting, caused by using a landscape page. The type area is malformed, but this is no problem if you only include external pages


% bibliography
% include biblatex, if you want to use it:
% \usepackage{biblatex}
\usepackage[backend=bibtex]{biblatex} % biber is recommended as a backend, but for simplicity it is changed here back to standart bibtex:
\addbibresource{bib/localBibliography.bib}

% a few more useful packages:
\usepackage{color}		% for colouring stuff
\usepackage{siunitx}	% units
\usepackage{listings}	% including formated code snippets
\usepackage{csvsimple}	% for importing CSV files
\usepackage{subfigure}	% for subfigures
\usepackage{soul}		% strikethrough text
\usepackage{lipsum}		% dummy text
\usepackage{amssymb}


\begin{document} %----------- beginning of document -----------------------------------------------

\pagenumbering{Roman}	% different numbering until first chapter
\maketitle				% using personal.tex
\declarationAuthorship

\begin{abstract}
	\lipsum[5-8]
\end{abstract}

\tableofcontents

\cleardoublepage % important when using double sided layout
\pagenumbering{arabic} % numbering in normal numbers

\chapter{Examples}
	\label{chap: one}
	{\color{red}red text} and {\color{blue}blue text} \\
	different subscripts: \normalsubscripts$R_t$ \upsubscripts$R_t$ \\
	using Units: $R=200\,\milli\ohm + \SI{0.34567453}{\volt\per\metre} - 5\,\si{\second\per\metre\squared}$ \\
	some information\cite{laboranleitung:physik}\\
	german number: $3,5$ english number: $3.5$\\ % note changes when using ngerman document option

	\section{using images}
	Images can just be imported and used in a float environment with a caption and a lable to reference it.
	\begin{figure}
		\includegraphics[width=.6\textwidth]{img/lorem-ipsum.jpg}
		\caption{just a random image}
	\end{figure}

	Plots can be created directly with latex. It is recommended to do this in subfiles and just import the finished PDF pages. This speed us
	compilertimes by a lot. You should not change the size of precompiled images to keep fontsizes consistent.
	\pagebreak
	\begin{figure}
		\includegraphics[page=2]{plt/examplePlot.pdf}
		\caption[centering]{a nice plot}
		\label{fig: plot1}
	\end{figure}
	\begin{figure}
		\includegraphics[page=1]{plt/examplePlot.pdf}
		\caption{a area plot}
		\label{fig: area}
	\end{figure}

	Circuit diagramms can also be created using a package called \lstinline{circuitikz}. It is also recommended to get familiar with Inkscape which
	has a very good export to latex feature, as you can see in \autoref{subfig: svg}.
	\begin{figure}
		\subfigure[a circuit diagramm]{\label{subfig: circuit}\includegraphics{crc/exampleCircuit.pdf}}
		\hspace{2cm}
		\caption{using two figures}
	\end{figure}

	Using Inkscape, you can create SVG-vector graphics and import them easily into Latex.
	\begin{figure}
		\def\svgwidth{0.3\textwidth} % define desired width
		\graphicspath{{svg/}} % Inkludepath needet for the inputed file, double curly brackets needed for unknown reason
		\input{svg/exampleSVG.pdf_tex} % input the Latexfile generated by inkscape
		\caption{A image created with Inkscape}
		\label{subfig: svg}
	\end{figure}

	\pagebreak
\section{demo nested listing}
	\begin{itemize}
		\item hallo
		\begin{itemize}
			\item temp
			\begin{itemize}
				\item temp
				\begin{itemize}
					\item temp
				\end{itemize}
			\end{itemize}
		\end{itemize}
	\end{itemize}


\section{using Units}
	For this the \lstinline{siunitx} package is used. It provides Macros for all units.
	\begin{equation}
		200\,\kg
	\end{equation}
	The space between a number and it's unit should be a protected half-space, which can be created in latex using \lstinline{\,} In the classfile
	siunits is set up to use a separate macro for each subunit, even for size-modifiers:
	\begin{equation}
		200\,\milli\metre \cdot 2\,\mega\volt
	\end{equation}
	Siunits also allows for reformatting of numbers as well as units. Use the \lstinline{\SI} and \lstinline{\si} macros for that:
	\begin{equation}
		e = \SI{1.602176634E-19}{\coulomb} % formats number + units, autospacing, transform to german comma
	\end{equation}
	\begin{equation}
		\SI[exponent-to-prefix]{1e-6}{\metre}% you can also autocalculate prefixes
	\end{equation}
	\begin{equation}
		124\,\si[per-mode=fraction]{\kilo\metre\per\second\squared} % only units
	\end{equation}
	\begin{equation}
		\num{0.0004}\,\lumen % only numbers,  \per and \squared do not work
	\end{equation}

\section{Using formulas}
	\label{sec: formula}
	a numberd formula:
	\begin{equation}
		\label{eq: einhalb} % always lable your stuff
		0,5=\frac{1}{3}
	\end{equation}
	\autoref{eq: einhalb} is nice, but how about multiple lines:
	\begin{equation}
	\begin{split} % you need do nest this
		x &= x^2+3 \\
		\Leftrightarrow 0 &= x^2-x+3 \\
	\end{split}
	\end{equation}
	and how could you align formulas?
	\begin{align}
		x_1 &= 6 &&|\;\mbox{mit } x \in \mathbb{N} \\
		x_2 &= 33+\abs{\frac{1}{4}} &&|\;x_1+3 \\
			&= 33,25 &&\mbox{| don't number everything} \notag \\
		x_3 &= 10^{22}
	\end{align}

\section{formating code}
	\label{sec: code}
	use the listings package:
	% gobble removes the whitespace that is only in the latex code to make it look organised
	\begin{lstlisting}[language=c,gobble=8]
		#include <stdlib.h>
		#include <sdtio.h>

		int main(void) {
			printf("Hello World");
			return 0;
		}
	\end{lstlisting}
	% or input from external file:
	%\lstinputlisting[language=c]{main.c}

\section{CSV files}
	\label{sec: messwerte}
	import a csv as table:\\
	\csvautotabular{csv/bsp.csv}\\
	or do it manually to get more control:
	\begin{table}
		\caption{a nice list of numbers}
		\begin{tabular}{c|c}
			first row & second row \\\hline\hline
			\csvreader[
				late after line=\\\hline,
				late after last line=\\\hline
			]{csv/bsp.csv}{}{number: $\csvcoli\,\metre$ & is not \csvcoliii}
		\end{tabular}
	\end{table}

\section{seperating the document}
	This was inputed from anothe file!!

\clearpage
\KOMAoptions{paper=landscape,pagesize,DIV=9} % rotate page to landscape mode (because why not XD), this causes a warning, use with cation
\recalctypearea
\chapter{attachment}
% manually include a PDF as not numbered section
\textbf{\Large{Messprotokoll oder so}} % just so it'S not empty
\phantomsection % Anker für den Hyperlink
\addcontentsline{toc}{section}{Messprotokoll} % add to table of content
\chaptermark{Messprotokoll}	% change headmark
%\includepdf[pages=-,pagecommand={},width=\paperwidth]{temp.pdf} % comment in to include pdf

As you can see its also possible to have some pages sideways. Just keep in mind you might need to adapt the margins

\newpage
\KOMAoptions{paper=portrait,pagesize} % and back
\recalctypearea

\printbibliography
\noindent\begin{minipage}{\textwidth} % prevent automatic pagebreaks
	\listoffigures
	\listoftables
\end{minipage}
\end{document}
