% \iffalse meta-comment
%
% Copyright (C) 2020 by Jan Wille <mail@janwille.de>
% -------------------------------------------------------
%
% This file may be distributed and/or modified under the
% conditions of the LaTeX Project Public License, either version 1.3
% of this license or (at your option) any later version.
% The latest version of this license is in:
%
%    http://www.latex-project.org/lppl.txt
%
% and version 1.3 or later is part of all distributions of LaTeX
% version 2021-01-01 or later.
%
% \fi
%
%<*article|report|standalone>
\typeout{Adapted from KOMA script for use in Hochschule Hannover <2022-03-28> Jan Wille}
\NeedsTeXFormat{LaTeX2e}
%
%<article>\newcommand{\myClassName}{HsH-article}
%<report>\newcommand{\myClassName}{HsH-report}
%<standalone>\newcommand{\myClassName}{HsH-standalone}
\ProvidesClass{\myClassName}[2022/03/28 v1.4 HsH-Class based on KOMA]

%% commands for HsH-logo
\newcommand{\HsHlogoPath}{HSH-Logo.pdf}
\newcommand{\HsHlogoPage}{1}
\newcommand{\includeHsHlogohere}[1][5cm]{\includegraphics[width=#1,page=\HsHlogoPage]{\HsHlogoPath}}
\newif\if@german%
%<article|report>\newif\if@todos\@todosfalse%

%% document options -------------------------------------------------------------------------------
\DeclareOption{german}{
  \@germantrue
}
\DeclareOption{ngerman}{
  \@germantrue
}
\DeclareOption{english}{
  \@germanfalse
}
\DeclareOption{f1}{
  \renewcommand{\HsHlogoPage}{2}
}
\DeclareOption{f2}{
  \renewcommand{\HsHlogoPage}{3}
}
\DeclareOption{f3}{
  \renewcommand{\HsHlogoPage}{4}
}
\DeclareOption{f4}{
  \renewcommand{\HsHlogoPage}{5}
}
\DeclareOption{f5}{
  \renewcommand{\HsHlogoPage}{6}
}
\DeclareOption{sans}{
  \renewcommand{\familydefault}{\sfdefault}
}
\DeclareOption{roman}{
  \renewcommand{\familydefault}{\rmdefault}
}
%<*article|report>
\DeclareOption{todos}{
  \@todostrue
}
%</article|report>
%<article>\DeclareOption*{\expandafter\PassOptionsToClass\expandafter{\CurrentOption}{scrartcl}}
%<report>\DeclareOption*{\expandafter\PassOptionsToClass\expandafter{\CurrentOption}{scrreprt}}
%<standalone>\DeclareOption*{\expandafter\PassOptionsToClass\expandafter{\CurrentOption}{standalone}}
%
%<article|report>\ExecuteOptions{fontsize=11pt,a4paper,ngerman,sans} % default options
%<standalone>\ExecuteOptions{margin=0.25cm,fontsize=11pt,a4paper,ngerman,sans} % default options
\ProcessOptions\relax
%% Load KOMA class
%<article>\LoadClass{scrartcl}
%<report>\LoadClass{scrreprt}
%<standalone>\LoadClass[multi=tikzpicture,multi=circuitikz]{standalone}
%<article|report>\KOMAoption{toc}{bibliography,listof}

%% package loading and config ---------------------------------------------------------------------
%
% inputencoding, utf8 recomendet
\PassOptionsToPackage{utf8}{inputenc}
\RequirePackage{inputenc}
%
% outputencoding, T1 recomendet
\PassOptionsToPackage{T1}{fontenc}
\RequirePackage{fontenc}
%
% font
\RequirePackage{lmodern}
%
% required for importing images
\RequirePackage{graphicx}
%
% a lot of nice Math
\RequirePackage{amsmath}
%
% german dezimal numbers
\if@german\RequirePackage{ziffer}\fi
%
%<*article|report>
% required for changing layout
\if@todos
  \PassOptionsToPackage{left=1cm,right=5cm,top=3cm,bottom=3cm}{geometry}
\else
  \PassOptionsToPackage{left=3cm,right=2.5cm,top=3cm,bottom=3cm}{geometry}
\fi
\RequirePackage{geometry}
%
% language specific localisation
\if@german\PassOptionsToPackage{ngerman}{babel}\else\PassOptionsToPackage{english}{babel}\fi
\RequirePackage{babel}
%
% required for header and footer
\PassOptionsToPackage{headsepline}{scrlayer-scrpage}
\RequirePackage{scrlayer-scrpage}
% compatability with listings
\RequirePackage{scrhack}
%
% references and lists are clikable
\PassOptionsToPackage{hidelinks}{hyperref}
\RequirePackage{hyperref}
%
% put anchors at the beginning of floats
\PassOptionsToPackage{hypcap=true}{caption}
\RequirePackage{caption}
%
% generat bookmarkes
\RequirePackage{bookmark}
%
% quotations
\PassOptionsToPackage{babel}{csquotes}
\if@german\PassOptionsToPackage{german=quotes}{csquotes}\fi
\RequirePackage{csquotes}
% this makes quotationmarks usable in text:
\MakeOuterQuote{"}
%
% Todonote package
% already changed the margins with geometry
\if@todos
  \PassOptionsToPackage{svgnames}{xcolor}
  \PassOptionsToPackage{
    textwidth=4.5cm,
    textsize=small,
    figwidth=.6\textwidth
  }{todonotes}
  \RequirePackage{todonotes}
  \setlength{\marginparwidth}{5.2cm}
\else
    \newcommand{\listoftodos}[1][]{}
    \newcommand{\todo}[2][]{}
    \newcommand{\missingfigure}[2][]{}
\fi
%
% foreach loops
\RequirePackage{pgffor}
%
%</article|report>
%
%% options for not loaded packages ----------------------------------------------------------------
%<article|report>\PassOptionsToPackage{backend=biber,style=alphabetic,sorting=nyt}{biblatex}
\PassOptionsToPackage{free-standing-units}{siunitx}
\PassOptionsToPackage{european,EFvoltages,straightvoltages,betterproportions}{circuitikz}

%
%<*article|report>
%% title matters ----------------------------------------------------------------------------------
%
% redefine title to accept a short tile (like chapter, etc..)
\newcommand*{\@shorttitle}{\@empty}
\renewcommand{\title}[2][]{
  \gdef\@title{#2}
  \gdef\@shorttitle{#1}
}
\newcommand*{\@matrikelnr}{}
\newcommand{\matrikelnr}[1]{\gdef\@matrikelnr{#1}}
\newcommand*{\@professor}{}
\newcommand{\professor}[1]{\gdef\@professor{#1}}
\newcommand*{\@keywords}{}
\newcommand{\keywords}[1]{\gdef\@keywords{#1}}
\let\@author\@empty
\let\@keywords\@empty

%</article|report>
%
%% typesetting indexes upright or standart
\def\subinrm#1{\sb{\mathrm{#1}}}
{\catcode`\_=13 \global\let_=\subinrm}
% the toggle for upright subscripts
\newcommand\upsubscripts{\catcode`\_=12 }
% the toggle for italic subscripts
\newcommand\normalsubscripts{\catcode`\_=8 }
% setting upright as default
\upsubscripts

%
%<*article|report>
%% header and footer
% first clear the default styles
\clearpairofpagestyles
% than we set the pagenumber for the plain and normal style
%<article>\cfoot*{\pagemark}
%<report>\ofoot*{\pagemark}
%
% now we define our custom header style
\newpagestyle{HsHheadings}{
% twosided left page header
  {
    \parbox[b]{\sls@headwidth}{
      \LaTeXraggedright
      \ifx\@shorttitle\@empty\@title\else\@shorttitle\fi
    }%
  }
% twosided right page header
  {
    \parbox[b]{\sls@headwidth}{
      \LaTeXraggedleft
      \leftmark
    }%
  }
% onesided page header
  {
    \parbox[b]{.45\sls@headwidth}{
      \LaTeXraggedright
      \ifx\@shorttitle\@empty\@title\else\@shorttitle\fi
    }%
    \hfill
    \parbox[b]{.45\sls@headwidth}{
      \LaTeXraggedleft
      \headmark
    }%
  }
  (\textwidth,.1pt)
}{
  {\pagemark}
  {\hfill\pagemark}
  {\hfill\pagemark}
}
\pagestyle{HsHheadings}
%<article>\automark{section}
%<report>\automark{chapter}
% Chaptertitles get same header/footer as everything:
%<report>\renewcommand*{\chapterpagestyle}{HsHheadings}

%% floats
% positioning of floats
\renewcommand{\fps@figure}{h!t}
\renewcommand{\fps@table}{h!t}
%
% check if KOMA is used
\ifx\KOMAScriptVersion\undefined\else
% center all floats
  \g@addto@macro\@floatboxreset\centering
% center all captions
  \setcapwidth{0.8\textwidth}
\fi
% abbreviations for names:
\if@german
  \renewcaptionname{ngerman}{\figurename}{Abb.}
  \renewcaptionname{ngerman}{\tablename}{Tab.}
  \renewcaptionname{ngerman}{\abstractname}{Abstract}
\else
  \renewcaptionname{english}{\figurename}{Fig.}
  \renewcaptionname{english}{\tablename}{Tab.}
\fi

%% styling for itemize
\renewcommand{\labelitemi}{$\bullet$}
\renewcommand{\labelitemii}{$\circ$}
\renewcommand{\labelitemiii}{$-$}
\renewcommand{\labelitemiv}{$\cdot$}

%% do stuff
% indentation of new paragraphs
\setlength\parindent{0pt}
% all stuff is pushed to the top of the page ingenerall
\raggedbottom
% rulers when using draft disabled
\renewcommand{\layercontentsmeasure}{\@empty}

%<*report>
% these commands are only defined for the book class, but they also make sens for the report class:
%% define easy shorthands to switch pagenumbering
\newif\if@mainmatter\@mainmattertrue
\newcommand*\frontmatter{%
  \if@twoside\cleardoubleoddpage\else\clearpage\fi
  \@mainmatterfalse\pagenumbering{Roman}%
}
\newcommand*\mainmatter{%
  \if@twoside\cleardoubleoddpage\else\clearpage\fi
  \@mainmattertrue\pagenumbering{arabic}%
}
\newcommand*\backmatter{%
  \if@openright\cleardoubleoddpage\else\clearpage\fi\@mainmatterfalse
}

%</report>
%% custom commands
% these two comand can convert decimal number into roman numerrals in etheir upper- or lowercase
\newcommand{\uproman}[1]{\uppercase\expandafter{\romannumeral#1}}
\newcommand{\lowroman}[1]{\romannumeral#1\relax}

%</article|report>
%
%% load config at beginning of document
\AtBeginDocument{
  \makeatletter
%<^*standalone>
% solve some common errors:
  \@ifpackageloaded{biblatex}{
% solves fontype warning in San-Serif
    \renewcommand*{\mkbibacro}[1]{\MakeUppercase{#1}}
  }{}
  \@ifpackageloaded{subfigure}{
    \newcommand{\subfigureautorefname}{\figureautorefname}
  }{}
%<^*standalone>
  \InputIfFileExists{config.tex}{
    \ClassInfo{\myClassName}{config.tex was found. Using it...}
  }{
    \ClassInfo{\myClassName}{no config.tex!! I hope you configered it yourself.}
  }
  %% provide some usefull commands:
% Makro for vertical lines around absolute values
  \providecommand{\abs}[1]{\ensuremath{\left\vert#1\right\vert}}
  \makeatother
}
%
%<*article|report>
%% nicer abstract ---------------------------------------------------------------------------------
\renewenvironment{abstract}{
  \quotation
  \setlength\parindent{0pt}
%<article>  \section*{\abstractname}
%<article>  \addcontentsline{toc}{section}{\abstractname}
%<report>  \chapter*{\abstractname}
%<report>  \addcontentsline{toc}{chapter}{\abstractname}
}{%
  \ifx\@keywords\@empty\else
%<article>    \vskip 0.5em
%<report>    \vskip 2em
    \paragraph{\if@german Schlüsselwörter:\else Keywords:\fi}
    \@keywords
  \fi
  \endquotation
}
%% decloration of Authorship ----------------------------------------------------------------------
\newif\if@oneAuthor \@oneAuthortrue
\if@german
  \setlocalecaption{ngerman}{dATitle}{Selbstständigkeitserklärung}
  \setlocalecaption{ngerman}{dAPronomen}{\if@oneAuthor ich\else wir\fi}
  \setlocalecaption{ngerman}{dAPluralendung}{\if@oneAuthor\else n\fi}
  \setlocalecaption{ngerman}{dAGenitiv}{\if@oneAuthor mir\else uns\fi}
  \setlocalecaption{ngerman}{dAText}{
    Hiermit bestätige\dAPluralendungname\space\dAPronomenname, dass die folgende Arbeit eigenständig von \dAGenitivname\space allein erstellt
    und unter Berücksichtigung der zur Verfügung gestellten Aufgabenstellung sowie dem Arbeitsmaterial unter Angabe aller verwendeten Quellen
    erarbeitet wurde. Die Regelungen und Konsequenzen eines Plagiats, inklusive disziplinarischer Maßnahmen, sind \dAGenitivname\space
    bewusst. Insbesondere wurden alle Zitate und gedanklichen Übernahmen als solche kenntlich gemacht.
  }
\else
  \setlocalecaption{english}{dATitle}{Declaration of Authorship}
  \setlocalecaption{english}{dAPronomen}{\if@oneAuthor I\else we\fi}
  \setlocalecaption{english}{dAPronomenf}{\if@oneAuthor I\else We\fi}
  \setlocalecaption{english}{dAPronomverb}{\if@oneAuthor am\else are\fi}
  \setlocalecaption{english}{dAGenitiv}{\if@oneAuthor my\else our\fi}
  \setlocalecaption{english}{dAText}{
    \dAPronomenfname\space hereby certify that the work \dAPronomenname\space\dAPronomverbname\space submitting is entirely of
    \dAGenitivname\space own making except where otherwise indicated. \dAPronomenfname\space\dAPronomverbname\space aware of regulations
    concerning plagiarism, including disciplinary actions that may result from it. Any use of the works of any other author, in any form, is
    properly acknowledged at their point of use.
  }
\fi
\newcommand\declarationAuthorship{
  \renewcommand{\thanks}{\sbox0}
  \foreach \x [count=\i] in \@author{\ifnum\i>1\global\@oneAuthorfalse\fi}
%
  \thispagestyle{plain}
  \vspace*{\fill}
  {\rule{\textwidth}{0.1pt}}
  \vskip 3em
  {\centering\huge\textbf{\dATitlename}\par}
  \vskip 3em
  {\dATextname\par}
% creating a signature line for every author
  {\raggedleft
    \foreach \x in \@author{
      \begin{tabular}[t]{r}
        \\[3em]
        \rule{4cm}{0.4pt}\\
        {\footnotesize\x}
      \end{tabular}
    }
  \par}
}
%
%% reconfig Titlepage -----------------------------------------------------------------------------
\newtoks\@tabtoks
\newcommand\addtabtoks[1]{\global\@tabtoks\expandafter{\the\@tabtoks#1}}
\newcommand\eaddtabtoks[1]{\edef\mytmp{#1}\expandafter\addtabtoks\expandafter{\mytmp}}
% %%\newcommand*\resettabtoks{\global\@tabtoks{}}
\newcommand*\printtabtoks{\the\@tabtoks}
\addtokomafont{publishers}{\normalsize}
%
%<article>\renewcommand\maketitle[1][c]{
%<report>\renewcommand\maketitle[1][l]{
  \expandafter\ifnum \csname scr@v@3.12\endcsname>\scr@compatibility\relax
  \else
    \def\and{%
      \end{tabular}
      \hskip 1em \@plus.17fil
      \begin{tabular}[t]{c}%
    }
  \fi
%<*article>
  \par
  \ifx\@uppertitleback\@empty\else
    \ClassWarning{\KOMAClassName}{%
      non empty \string\uppertitleback\space ignored
      by \string\maketitle\MessageBreak
      in `titlepage=false' mode%
    }
  \fi
  \ifx\@lowertitleback\@empty\else
    \ClassWarning{\KOMAClassName}{%
      non empty \string\lowertitleback\space ignored
      by \string\maketitle\MessageBreak
      in `titlepage=false' mode%
    }
  \fi
%</article>
%<report>  \begin{titlepage}
%<article>  \begingroup
    \let\@param#1
    \ifx\@param\@empty
      \ClassError{\myClassName}{\maketitle\space with empty option}{
        \maketitle[] has been called with an empty parameter, this doesn't work. Use \maketitle instead.
      }
    \fi
%<*report>
    \if@titlepageiscoverpage
      \edef\titlepage@restore{
        \noexpand\endgroup
        \noexpand\global\noexpand\@colht\the\@colht
        \noexpand\global\noexpand\@colroom\the\@colroom
        \noexpand\global\vsize\the\vsize
        \noexpand\global\noexpand\@titlepageiscoverpagefalse
        \noexpand\let\noexpand\titlepage@restore\noexpand\relax
      }
      \begingroup
      \topmargin=\dimexpr \coverpagetopmargin-1in\relax
      \oddsidemargin=\dimexpr \coverpageleftmargin-1in\relax
      \evensidemargin=\dimexpr \coverpageleftmargin-1in\relax
      \textwidth=\dimexpr
      \paperwidth-\coverpageleftmargin-\coverpagerightmargin\relax
      \textheight=\dimexpr
      \paperheight-\coverpagetopmargin-\coverpagebottommargin\relax
      \headheight=0pt
      \headsep=0pt
      \footskip=\baselineskip
      \@colht=\textheight
      \@colroom=\textheight
      \vsize=\textheight
      \columnwidth=\textwidth
      \hsize=\columnwidth
      \linewidth=\hsize
    \else
      \let\titlepage@restore\relax
    \fi
    \let\footnotesize\small
    \let\footnoterule\relax
    \let\footnote\thanks
%</report>
%<article>    \let\titlepage@restore\relax
    \renewcommand*\thefootnote{\@fnsymbol\c@footnote}%
    \let\@oldmakefnmark\@makefnmark
    \renewcommand*{\@makefnmark}{\rlap\@oldmakefnmark}%
%<article>    \next@tdpage
    \ifx\@extratitle\@empty
%<article>      \ifx\@frontispiece\@empty\else \mbox{}\fi
%<*report>
      \ifx\@frontispiece\@empty\else
        \if@twoside\mbox{}\next@tpage\fi
        \noindent\@frontispiece\next@tdpage
      \fi
%</report>
    \else
%<article>      \@makeextratitle
%<*report>
      \noindent\@extratitle
      \ifx\@frontispiece\@empty
      \else
        \next@tpage
        \noindent\@frontispiece
      \fi
      \next@tdpage
%</report>
    \fi
%<*article>
    \ifx\@frontispiece\@empty
      \ifx\@extratitle\@empty\else\next@tdpage\fi
    \else
      \next@tpage
      \@makefrontispiece
      \next@tdpage
    \fi
    \global\@topnum=\z@
%</article>
    \setparsizes{\z@}{\z@}{\z@\@plus 1fil}\par@updaterelative
    \IfFileExists{\HsHlogoPath}{}{
      \ClassError{\myClassName}{HsH-Logo.pdf not found!}{
        The HsH Logo is necasary for the titlepage! Try putting it next to your source file or use \HsHlogoPath to define the file location
      }
    }
    \vspace*{1cm}
    \begin{minipage}[t]{\textwidth}%
      \ifx\@titlehead\@empty \else
        \usekomafont{titlehead}{\@titlehead}%
      \fi
      \hfill
% image with referrencepoint in lower left corner:
      \raisebox{0pt}[\ht\strutbox][\dp\strutbox]{\includeHsHlogohere}
    \end{minipage}
    \raisebox{10pt}{\rule{\textwidth}{0.5pt}}
    \null
%<article>    \vskip 2em
%<report>    \vfill
    \begingroup
      \if\@param c\centering\fi
      \if\@param r\raggedleft\fi
      \ifx\@subject\@empty\else
        {\usekomafont{subject}{\@subject\par}}
%<article>        \vskip 1.5em
%<report>        \vskip 3em
      \fi
      {\usekomafont{title}{\huge\@title\par}}
%<article>      \vskip .5em
%<report>      \vskip 1em
      {\ifx\@subtitle\@empty\else\usekomafont{subtitle}\@subtitle\par\fi}
%<article|report>      \vskip 4em
      {\ifx\@matrikelnr\@empty
        \if\@author\@empty\else\usekomafont{author}{
          \parbox{\dimexpr\linewidth}{
            \if\@param c\centering\fi
            \if\@param r\raggedleft\fi
            \@author
          }
        }\fi
      \else
        \if\@author\@empty\else
          % sneeky comma needed after \@matrikelnr to deal with single item lists
          \foreach \x [count=\i,evaluate=\i as \y using {{\@matrikelnr,}[\i-1]}] in \@author {\eaddtabtoks{\x & \y\protect\\}}
          \usekomafont{author}{\def\arraystretch{1.2}
            \if\@param l\begin{tabular}{@{}l l}\printtabtoks\end{tabular}\fi
            \if\@param c\begin{tabular}{l l}\printtabtoks\end{tabular}\fi
            \if\@param r\begin{tabular}{r r@{}}\printtabtoks\end{tabular}\fi
          }
        \fi
      \fi}
%<article>      \vskip 1em
%<report>      \vskip 1.5em
      {\usekomafont{date}{\@date\par}}
%<article>      \vskip 1em
%<report>      \vskip \z@ \@plus3fill
     {\if\@professor\@empty\else\usekomafont{publishers}{Professor\if@german(in)/Lehrbeauftragte(r)\fi: \@professor}\fi}
%<*article>
      \ifx\@dedication\@empty\else
        \vskip 2em
        {\usekomafont{dedication}{\@dedication \par}}%
      \fi
%</article>
      \par
    \endgroup
%<article>    \vskip 2em
%<report>    \vskip 3em
%<article>    \ifx\titlepagestyle\@empty\else\thispagestyle{\titlepagestyle}\fi
    \@thanks\global\let\@thanks\@empty
%<*report>
    \vfill\null
    \if@twoside
      \@tempswatrue
      \expandafter\ifnum \@nameuse{scr@v@3.12}>\scr@compatibility\relax
      \else
        \ifx\@uppertitleback\@empty\ifx\@lowertitleback\@empty
          \@tempswafalse
        \fi\fi
      \fi
      \if@tempswa
        \next@tpage
        \begin{minipage}[t]{\textwidth}
          \@uppertitleback
        \end{minipage}\par
        \vfill
        \begin{minipage}[b]{\textwidth}
          \@lowertitleback
        \end{minipage}\par
        \@thanks\global\let\@thanks\@empty
      \fi
    \else
      \ifx\@uppertitleback\@empty\else
        \ClassWarning{\KOMAClassName}{%
          non empty \string\uppertitleback\space ignored
          by \string\maketitle\MessageBreak
          in `twoside=false' mode%
        }
      \fi
      \ifx\@lowertitleback\@empty\else
        \ClassWarning{\KOMAClassName}{%
          non empty \string\lowertitleback\space ignored
          by \string\maketitle\MessageBreak
          in `twoside=false' mode%
        }
      \fi
    \fi
    \ifx\@dedication\@empty
    \else
      \next@tdpage\null\vfill
      {\centering\usekomafont{dedication}{\@dedication \par}}%
      \vskip \z@ \@plus3fill
      \@thanks\global\let\@thanks\@empty
      \cleardoubleemptypage
    \fi
    \ifx\titlepage@restore\relax\else\clearpage\titlepage@restore\fi
%</report>
%<article>  \endgroup
%<report>  \end{titlepage}
  \setcounter{footnote}{0}
  \expandafter\ifnum \csname scr@v@3.12\endcsname>\scr@compatibility\relax
    \let\thanks\relax
    \let\maketitle\relax
    \let\@maketitle\relax
    \global\let\@thanks\@empty
    \global\let\@author\@empty
    \global\let\@date\@empty
    \global\let\@title\@empty
    \global\let\@subtitle\@empty
    \global\let\@extratitle\@empty
    \global\let\@frontispiece\@empty
    \global\let\@titlehead\@empty
    \global\let\@subject\@empty
    \global\let\@publishers\@empty
    \global\let\@uppertitleback\@empty
    \global\let\@lowertitleback\@empty
    \global\let\@dedication\@empty
    \global\let\@matrikelnr\@empty
    \global\let\@prof\@empty
    \global\let\author\relax
    \global\let\title\relax
    \global\let\extratitle\relax
    \global\let\titlehead\relax
    \global\let\subject\relax
    \global\let\publishers\relax
    \global\let\uppertitleback\relax
    \global\let\lowertitleback\relax
    \global\let\dedication\relax
    \global\let\date\relax
    \global\let\matrikelnr\relax
    \global\let\professor\relax
  \fi
  \global\let\and\relax
}
%</article|report>
%</article|report|standalone>
%
% Config file -------------------------------------------------------------------------------------
%<*config>
%% Meta Data for the PDF file
\@ifpackageloaded{hyperref}{
  \hypersetup{
    pdfinfo={
      Title={\@title},
      Author={\@author},
      Subject={\@subject},
      Keywords={\@keywords}
    }
  }
}{}
%% configs for optional packages:
\@ifpackageloaded{biblatex}{
  \ExecuteBibliographyOptions{
    bibwarn=true,
    url=true,
    isbn=false,
  }
}{}
\@ifpackageloaded{todonotes}{
  \setuptodonotes{
    backgroundcolor=orange!60!white,
    linecolor=orange,
    bordercolor=white
  }
}{}
\@ifpackageloaded{siunitx}{
  \if@german
    \sisetup{output-decimal-marker={,}} % use comma as dezimal
  \fi
  \sisetup{
    tight-spacing=true,
    per-mode=symbol,
    scientific-notation=engineering,
    %exponent-to-prefix, % use this instead of sci-notatio
    round-mode = places, % round numbers
    round-precision = 3, % to 3 places
  }
}{}
\@ifpackageloaded{listings}{
  \PassOptionsToPackage{svgnames}{xcolor}
  \RequirePackage{xcolor}
  \RequirePackage{lstautogobble}
  \lstset{
    commentstyle=\color{gray},
    keywordstyle=\color{FireBrick}\bfseries,
    stringstyle=\color{DodgerBlue},
    showstringspaces=false,
    basicstyle=\ttfamily,
    tabsize=4,
    literate={Ö}{{\"O}}1{Ä}{{\"A}}1{Ü}{{\"U}}1{ß}{{\ss}}1{ü}{{\"u}}1{ä}{{\"a}}1{ö}{{\"o}}1,
    escapeinside={(*@}{@*)},
    autogobble=true,
  }
}{}
\@ifpackageloaded{enumitem}{
  \RequirePackage{amssymb}
  \newlist{checklist}{itemize}{2}
  \setlist[checklist]{label=$\square$}
}{}
\@ifpackageloaded{csvsimple}{
  \csvstyle{every csv}{separator=semicolon} % global seperator for csv files
}{}
\@ifpackageloaded{circuitikz}{
  \ctikzset{
    font=\footnotesize,
    bipole annotation style/.style={font=\tiny ,inner sep=1pt},
  }
}{}
\@ifpackageloaded{pgfplots}{
  \usepgfplotslibrary{fillbetween} % mark areas under or between graphs
  \usetikzlibrary{patterns} % for using pattern in plots
  \if@german
    \pgfplotsset{/pgf/number format/use comma}
  \fi
  \pgfplotsset{
    compat=1.16,
    /pgf/number format/read comma as period,
    every tick label/.append style={font=\tiny},
    every axis legend/.append style={font=\footnotesize},
    MyPlots/.style = {		% Style for all plots
      width=.6\textwidth,
      grid=major,
      grid style={dashed,gray!50},
      legend pos=outer north east,
      FM1/.style = {blue, thick, mark=x, mark size=1.5,samples=100},  	% Format 1
      FM2/.style = {orange, thick, mark=*, mark size=1,samples=100}, 		% Format 2
      FM3/.style = {green, thick, mark=square*, mark size=1,samples=100}, % Format 3
      FM4/.style = {red, thick, mark=diamond*, mark size=1,samples=100}, 	% Format 4
      FM5/.style = {pink, thick, mark=triangel*, mark size=1,samples=100},% Format 5
      errBars/.style = {error bars/.cd, y dir=both, y explicit, % ermöglicht Fehlerbalken
              error mark options={mark size=1pt,rotate=90}},
    }
  }
}{}
\@ifpackageloaded{pgfplotstable}{
  \pgfplotstableset{
    col sep=semicolon, % global seperator for csv files
  }
}{}
%</config>
%
% \Finale
\endinput
