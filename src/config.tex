% abbreviations for names:
\if@german
	\renewcommand{\abstractname}{Abstract} % Abstract
	\renewcaptionname{ngerman}{\figurename}{Abb.} %Figure
	\renewcaptionname{ngerman}{\tablename}{Tab.} %Table
\else
	\renewcaptionname{english}{\figurename}{Fig.} %Figure
	\renewcaptionname{english}{\tablename}{Tab.} %Table
\fi

% center all floats:
\ifx\KOMAScriptVersion\undefined\else % check if KOMA is used
	\g@addto@macro\@floatboxreset\centering % center all floats
	\setcapwidth[c]{0.8\textwidth} % center all captions
	%\setcapwidth[l]{0.8\textwidth} % leftalign all captions
\fi

% Meta Data for the PDF file using values from personal.tex:
\@ifpackageloaded{hyperref}{
	\hypersetup{
		pdfinfo={
			Title={\@title},
			Author={\@author},
			Subject={\@subject},
			Keywords={\@keywords}
		}
	}
}{}%
% configs for optional packages:
\@ifpackageloaded{biblatex}{
	\ExecuteBibliographyOptions{
		bibwarn=true,
		url=true,
		isbn=false,
	}
}{}%
\@ifpackageloaded{siunitx}{
	\if@german % if german
		\sisetup{output-decimal-marker={,}} % use comma as dezimal
	\fi
	\sisetup{
		tight-spacing=true,
		per-mode=symbol,
		scientific-notation=engineering,
		% exponent-to-prefix, % use this instead of sci-notatio
		round-mode = places, % round numbers
		round-precision = 3, % to 3 places
	}
}{}%
\@ifpackageloaded{listings}{
	\lstset{
		commentstyle=\color{gray},
		keywordstyle=\color{red},
		stringstyle=\color{blue},
		showstringspaces=false,
		basicstyle=\ttfamily,
		tabsize=4,
		literate={Ö}{{\"O}}1{Ä}{{\"A}}1{Ü}{{\"U}}1{ß}{{\ss}}1{ü}{{\"u}}1{ä}{{\"a}}1{ö}{{\"o}}1,
		escapeinside={(*@}{@*)},
	}
}{}%
\@ifpackageloaded{enumitem}{
	\usepackage{amssymb}
	\newlist{checklist}{itemize}{2}
	\setlist[checklist]{label=$\square$}
}{}%
\@ifpackageloaded{csvsimple}{
	\csvstyle{every csv}{separator=semicolon}
}{}%
\@ifpackageloaded{circuitikz}{
	\ctikzset{
		font=\footnotesize,
		bipole annotation style/.style={font=\tiny ,inner sep=1pt},
	}
}{}%
\@ifpackageloaded{pgfplots}{
	\usepgfplotslibrary{fillbetween} % mark areas under or between graphs
	\usetikzlibrary{patterns} % for using pattern in plots
	\if@german
		\pgfplotsset{/pgf/number format/use comma}
	\fi
	\pgfplotsset{
		compat=1.16,
		/pgf/number format/read comma as period,
		every tick label/.append style={font=\tiny},
		every axis legend/.append style={font=\footnotesize},
		MyPlots/.style = {		% Style für alle Plots
			width=.6\textwidth,
			grid=major,												% Gitter für haupt Ticks
			grid style={dashed,gray!50},							% Hintergrund Gitter
			legend pos=outer north east,							% Position der Legende
			FM1/.style = {blue, thick, mark=x, mark size=1.5,samples=100},  	% Format 1
			FM2/.style = {orange, thick, mark=*, mark size=1,samples=100}, 		% Format 2
			FM3/.style = {green, thick, mark=square*, mark size=1,samples=100}, % Format 3
			FM4/.style = {red, thick, mark=diamond*, mark size=1,samples=100}, 	% Format 4
			FM5/.style = {pink, thick, mark=triangel*, mark size=1,samples=100},% Format 5
			errBars/.style = {error bars/.cd, y dir=both, y explicit, % ermöglicht Fehlerbalken
							error mark options={mark size=1pt,rotate=90}},
		}
	}
}{}%
\@ifpackageloaded{pgfplotstable}{
	\pgfplotstableset{
		col sep=semicolon, % global seperator for csv files
	}
}{}%
