\chapter{Formatting text} \label{:chap formating}
	To begin I want to show you the basics of how to get text onto the page and structure it. You can also see how I created this exact text as an
	example.

	\section{Texts and paragraphs} \label{sec: text and par}
		Writing text for a LaTeX document is very easy. You just put the text in. LaTeX doesn't care about line breaks or whitespace, so its up to
		your preferences how the source code looks. You could just write the whole document as one super long line, but it makes sense to break it up
		and keep it readable. One common way is to put every sentence on a new line. Alternatively lots of editors can break lines for you after you
		reach a certain width (that's how this source code is formatted).

		LaTeX will automatically space and break your text to optimal use the available space while still looking good. You can however assist it when
		it struggles. Putting hyphens (-) into a word tells LaTeX to break it there. If it's a word you use a lot, you can use
		|\hyphenation{very-long-word}| in your preamble to tell LaTeX how to split it everywhere.

		Lots of examples will tell you that |\\| is a line break. While this is correct you shouldn't use it to break your text block and
		start a new line. A block of text is a paragraph and should be ended with the |\par| command. For ease of use LaTeX will
		automatically use this command if you leave a blank line. (see this source code)

		If you want to further separate paragraphs visually (when you finish a train of thought for example) you can use the commands
		|\smallskip|, |\medskip| and |\bigskip|

	\section{Headings} \label{sec: headings}
		The exact commands available will vary depending on you your documentclass, but they will always be a single command that excepts any text
		inside curly brackets, for example |\section{text}|. The different commands form a hierarchy you can nest into each other, keeping
		track of its parent element. That means you don't have to worry about any formatting or numbering, LaTeX will handle that for you.

		When using an \emph{article} documentclass, the commands available are |\section{}|, |\subsection{}| and
		|\subsubsection{}|. Should you need more nesting levels, you are usually overcomplicating things, but you could additionally use the
		|\paragraph{}| command, which gives you a slightly bigger, bold first word for your paragraph.

		The \emph{report} documentclass adds the additional command |\chapter{}| as the highest heading level. You can still use the previous
		three commands for the nested headings. A chapter automatically starts on a new page, so it should be at leas two pages long. You also get the
		command |\part{}|, which creates a separate page for the part's title. These should only be used in very long documents.

	\section{Text spacing} \label{sec: spacing}
		By default, these classes add no spacing between paragraph, but sometimes you want to visually enforce a breakpoint in your argumentation. For
		that you can add some space in between to paragraph by using one of the commands |\bigskip|, |\medskip| or
		|\smallskip|. How much space you want depends on your tase, but you should keep it consistent. Here is an example:

		\bigskip
		This text has a big space before it,

		\medskip
		Here I used just some medium spacing

		\smallskip
		and this is a small space.

	\section{Breaking pages} \label{sec: pagebreak}
		Sometimes you will find yourself in situations, where you don't like where LaTeX splits your text to the next page. So first, take some
		advice: Don't worry about it for now. Your text will probably change a few times before its final. Just leave it.

		If you are at the final stage, you can do a beautifying pass. Now you can use |\pagebreak| to tell LaTeX about better places to
		break the text.

		Should you still not be happy (this happens especially with multiple images/tables in close proximity) you most likely have to little text and
		should redesign your document. But if you absolutely want to print it that way, you can use |\clearpage| to force all
		figures/tables to be put onto the page and then start a new page.

		\medskip
		You might also need just a little more space only a page to just fit one more sentences. For that you can use the command
		|\enlargethispage{N\baselineskip}| with $N$ being the number of lines you need. Use this sparingly however, as the bottom margin is
		there for a reason and you shouldn't intrude on the footer too much.


	\section{Text styling} \label{sec: styling}
		When writing text, you will need to \emph{emphasize} certain parts of the text. The easiest way is to use the |\emph{}| command
		around you text. You can also nest it \emph{to \emph{emphasize} even more}.

		If you want to change to a specific font-type, you can do that like this:

		\smallskip
		\begin{tabular}{l l}
			|\underline{text}| & \underline{Underlined} \\
			|\textbf{text}| & \textbf{Bold Font} \\
			|\textii{text}| & \textit{Italic Font} \\
			|\textrm{text}| & \textrm{Roman Font} \\
			|\texttt{text}| & \texttt{Typewriter Font} \\
			|\textsc{text}| & \texttt{Small Caps Font} \\
		\end{tabular}

		\medskip
		You might also want to change your text colour, which is what the {color} package is for. It provides the
		|textcolor{colour}{text}| command, \textcolor{red}{which allows you} \textcolor{blue}{to change your text colour}.

		\section{Special characters} \label{sec: special-charaters}
			\subsection{LaTeX command characters}
				As in most programming languages, some characters are used for LaTeXes commands and can't be used in text directly. Here is a table
				explaining them all:

				\smallskip
				\begin{tabular}{l l l}
					\emph{character} & \emph{special meaning} & \emph{how to get character} \\
					\textbackslash & beginning of a command & |\textbackslash| \\
					\{ and \} & denote a code block & |\{| and |\}|\\
					\% & beginning of a comment & |\%| \\
					\# & macro parameter character & |\#| \\
					\$ & beginning/end of math mode & |\$| \\
					\textasciitilde & non-breaking space & |\textasciitilde| \\
					\emph{only inside math mode:} \\
					$\_$ & subscript & |\_| \\
					\textasciicircum & superscript & |\textasciicircum| \\
				\end{tabular}

			\subsection{Invisible characters}
				To properly typeset your text you may need a number of special characters under specific circumstances:

				\smallskip
				\begin{tabular}{l l l}
					\emph{explanation} & \emph{command} & \emph{example} \\
					non-breaking space & |~| & Max~Mustermann (Names shouldn't be broken) \\
					3/4 non-breaking space & |\;| & 10\;000 (separate thousands)\\
					medium non-breaking space & |\:| & z.\:B. (abbreviations) \\
					1/2 non-breaking space & |\,| & 1\,V (number + unit) \\
					normal space & |\space| & in case some command eats up all your space \\
				\end{tabular}
