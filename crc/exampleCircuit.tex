\documentclass[11pt,ngerman]{HsH-standalone}

\usepackage{siunitx} % kennt SI einheiten
\usepackage{amsmath} % schöne Formeln
\usepackage{circuitikz} % zeichnen von Schaltplänen

\begin{document} %----------------------Inhalt---------------------------------------------------%

\begin{circuitikz} \draw
	(0,0)	to[V, v<=$U_q$, i=$I_q$]  ++(0,2)
			-- ++(2,0)
			to[R, i^=$I_R$]	++(0,-2)
			to[rmeter, t=A] ++(-2,0)
	;
\end{circuitikz}

\begin{circuitikz} \draw
	(0,0)	to[I, v^<=$U_q$, i=$I_q$]  ++(0,2)
			-- ++(2,0) coordinate(a)
			to[R, *-*, v^=$U_R$] ++(0,-2) coordinate(b)
			-- ++(-2,0)
	(a) 	-- ++(1.5,0)
			to[rmeterwa, t=V] ++(0,-2)
			-- (b)
	;
\end{circuitikz}

\begin{circuitikz} \draw
	(0,-2)	coordinate(home)
			to[V, v<=$U_q$] ++(0,2)
	;
	\foreach \i in {0,...,3} \draw
		(0,0) -- ++(1.5+\i,0)
			to[R, *-*,l=$R_\i$] ++(0,-2)
			-- (home)
	;
\end{circuitikz}

\end{document}
